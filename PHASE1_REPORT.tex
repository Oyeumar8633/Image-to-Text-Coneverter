\documentclass[12pt,a4paper]{article}
\usepackage[utf8]{inputenc}
\usepackage[margin=1in]{geometry}
\usepackage{hyperref}
\usepackage{listings}
\usepackage{xcolor}
\usepackage{enumitem}
\usepackage{graphicx}
\usepackage{fancyhdr}
\usepackage{titlesec}

% Page style
\pagestyle{fancy}
\fancyhf{}
\fancyhead[L]{Phase 1 Report: Image-to-Word Converter}
\fancyhead[R]{\today}
\fancyfoot[C]{\thepage}

% Title formatting
\titleformat{\section}{\large\bfseries}{\thesection}{1em}{}
\titleformat{\subsection}{\normalsize\bfseries}{\thesubsection}{1em}{}

% Code listing style
\lstset{
    basicstyle=\ttfamily\small,
    breaklines=true,
    frame=single,
    backgroundcolor=\color{gray!10}
}

% Hyperref setup
\hypersetup{
    colorlinks=true,
    linkcolor=blue,
    filecolor=magenta,
    urlcolor=cyan,
    pdftitle={Phase 1 Report: Image-to-Word Converter},
    pdfauthor={Development Team}
}

\begin{document}

\title{\textbf{Phase 1 Report: Image-to-Word Converter MVP}}
\author{Muhammad Umar Iftikhar 21i-2710, Muhammad Muneer 22i-0526}
\date{\today}
\maketitle

\section{Project Overview}

This document outlines the implementation of Phase 1 of the Image-to-Word Converter project. The MVP is a Streamlit-based web application that extracts text from images using OCR technology and preserves formatting (Bold, Italic, Alignment) in Microsoft Word (.docx) format.

\section{Technical Stack}

\begin{itemize}
    \item \textbf{Frontend Framework}: Streamlit 1.28.0+
    \item \textbf{OCR Engines}: 
    \begin{itemize}
        \item \textbf{PaddleOCR 3.4.0+} (Primary - Best overall accuracy for printed and handwritten text)
        \item \textbf{EasyOCR 1.7.0+} (Alternative - Good for handwriting)
        \item \textbf{Tesseract OCR} (via pytesseract) - Fast for printed text
    \end{itemize}
    \item \textbf{Document Generation}: python-docx 1.1.0+
    \item \textbf{Image Processing}: OpenCV 4.8.0+ and Pillow 10.0.0+
    \item \textbf{Deep Learning}: PyTorch 2.0.0+ (for EasyOCR and PaddleOCR)
    \item \textbf{Language}: Python 3.8+
\end{itemize}

\section{Project Structure}

\begin{verbatim}
PPIT Project/
├── app.py                 # Main Streamlit application
├── ocr_engine.py          # OCR processing and formatting detection
├── requirements.txt       # Python dependencies
├── .gitignore            # Git ignore file (excludes venv/)
├── venv/                 # Virtual environment (created locally)
└── PHASE1_REPORT.md       # This documentation file
\end{verbatim}

\subsection{OCR Engine Support}

The application supports \textbf{three OCR engines}:

\begin{enumerate}
    \item \textbf{PaddleOCR} (Recommended)
    \begin{itemize}
        \item Best overall accuracy for both printed and handwritten text
        \item Deep learning-based with excellent recognition rates
        \item Automatic angle detection and document preprocessing
        \item Models downloaded automatically on first use
    \end{itemize}
    
    \item \textbf{EasyOCR}
    \begin{itemize}
        \item Excellent for handwriting recognition
        \item Deep learning-based
        \item Good for complex layouts
        \item Models downloaded automatically on first use
    \end{itemize}
    
    \item \textbf{Tesseract OCR}
    \begin{itemize}
        \item Fast and lightweight
        \item Best for clear printed documents
        \item Multiple PSM modes for different document types
        \item Requires system-level Tesseract installation
    \end{itemize}
\end{enumerate}

\section{Implementation Details}

\subsection{OCR Engine (\texttt{ocr\_engine.py})}

The OCR engine module provides the core functionality for text extraction and formatting detection.

\subsubsection{Key Components:}

\textbf{Image Preprocessing (\texttt{preprocess\_image})}
\begin{itemize}
    \item Converts images to grayscale
    \item Applies adaptive thresholding for binary conversion
    \item Implements denoising using OpenCV's fastNlMeansDenoising
    \item Optimizes images for better OCR accuracy
\end{itemize}

\textbf{Formatting Detection (\texttt{detect\_formatting\_from\_confidence})}
\begin{itemize}
    \item Analyzes OCR confidence scores to identify bold text
    \item Uses word height characteristics for formatting inference
    \item Implements heuristics for italic text detection
    \item Note: Formatting detection is probabilistic and works best with clear, well-formatted documents
\end{itemize}

\textbf{Alignment Detection (\texttt{detect\_alignment})}
\begin{itemize}
    \item Analyzes text block positions relative to page width
    \item Classifies alignment as left, center, or right
    \item Uses margin-based detection (20\% margins on each side)
\end{itemize}

\textbf{Text Extraction (\texttt{extract\_text\_with\_formatting})}
\begin{itemize}
    \item Uses Tesseract's \texttt{image\_to\_data} for detailed OCR information
    \item Groups words into paragraphs based on vertical positioning
    \item Preserves line breaks and paragraph structure
    \item Returns structured \texttt{TextBlock} objects with formatting metadata
\end{itemize}

\textbf{Data Structure (\texttt{TextBlock})}
\begin{itemize}
    \item Dataclass representing text blocks with:
    \begin{itemize}
        \item Text content
        \item Bold/Italic flags
        \item Confidence score
        \item Position and dimensions
        \item Alignment information
    \end{itemize}
\end{itemize}

\subsection{Streamlit Application (\texttt{app.py})}

The main application provides a user-friendly interface for the OCR conversion process.

\subsubsection{Features:}

\textbf{User Interface}
\begin{itemize}
    \item Clean, modern Streamlit UI with wide layout
    \item Image uploader supporting JPG, JPEG, and PNG formats
    \item Side-by-side image preview and processing options
    \item Real-time processing status indicators
\end{itemize}

\textbf{Functionality}
\begin{itemize}
    \item Image upload and preview
    \item Configurable preprocessing (enable/disable)
    \item Configurable formatting detection (enable/disable)
    \item One-click conversion to Word document
    \item Direct download of generated .docx file
    \item Extracted text preview before download
\end{itemize}

\textbf{Document Generation}
\begin{itemize}
    \item Creates formatted Word documents using python-docx
    \item Preserves bold and italic formatting
    \item Maintains paragraph alignment (left, center, right)
    \item Adds appropriate spacing between paragraphs
    \item Fallback to simple text extraction if formatting detection fails
\end{itemize}

\textbf{Error Handling}
\begin{itemize}
    \item Comprehensive exception handling
    \item User-friendly error messages
    \item Graceful fallback mechanisms
\end{itemize}

\subsection{Dependencies (\texttt{requirements.txt})}

All required Python packages with minimum version specifications:
\begin{itemize}
    \item \texttt{streamlit}: Web application framework
    \item \texttt{pytesseract}: Python wrapper for Tesseract OCR
    \item \texttt{python-docx}: Word document generation
    \item \texttt{opencv-python}: Image processing
    \item \texttt{Pillow}: Image manipulation
    \item \texttt{numpy}: Numerical operations
    \item \texttt{easyocr}: Deep learning OCR engine (optional, for handwriting)
    \item \texttt{torch}: PyTorch deep learning framework (for EasyOCR)
    \item \texttt{paddlepaddle}: PaddlePaddle deep learning framework (for PaddleOCR)
    \item \texttt{paddleocr}: PaddleOCR engine (recommended, best accuracy)
\end{itemize}

\section{Usage Instructions}

\subsection{Installation}

\begin{enumerate}
    \item \textbf{Install Tesseract OCR} (system-level dependency, optional if using only PaddleOCR/EasyOCR):
    \begin{itemize}
        \item \textbf{macOS}: \texttt{brew install tesseract}
        \item \textbf{Linux}: \texttt{sudo apt-get install tesseract-ocr}
        \item \textbf{Windows}: Download from \url{https://github.com/UB-Mannheim/tesseract/wiki}
        \item \textbf{Note}: Tesseract is optional if you only use PaddleOCR or EasyOCR
    \end{itemize}
    
    \item \textbf{Create and activate virtual environment} (recommended):
    \begin{lstlisting}[language=bash]
# Create virtual environment
python3 -m venv venv

# Activate virtual environment
# On macOS/Linux:
source venv/bin/activate
# On Windows:
venv\Scripts\activate
    \end{lstlisting}
    
    \item \textbf{Install Python dependencies}:
    \begin{lstlisting}[language=bash]
pip install --upgrade pip
pip install -r requirements.txt
    \end{lstlisting}
    
    \textbf{Note}: First-time installation will download deep learning models:
    \begin{itemize}
        \item PaddleOCR models (\textasciitilde100MB) - downloaded automatically on first use
        \item EasyOCR models (\textasciitilde100MB) - downloaded automatically on first use
        \item Models are cached locally for future use
    \end{itemize}
\end{enumerate}

\subsection{Running the Application}

\begin{enumerate}
    \item \textbf{Activate virtual environment} (if not already activated):
    \begin{lstlisting}[language=bash]
# On macOS/Linux:
source venv/bin/activate
# On Windows:
venv\Scripts\activate
    \end{lstlisting}
    
    \item \textbf{Start the Streamlit server}:
    \begin{lstlisting}[language=bash]
streamlit run app.py
    \end{lstlisting}
    
    \item \textbf{Access the application}:
    \begin{itemize}
        \item The application will open automatically in your default web browser
        \item Default URL: \texttt{http://localhost:8501}
    \end{itemize}
    
    \item \textbf{Using the application}:
    \begin{itemize}
        \item Upload a JPG or PNG image file
        \item Preview the uploaded image
        \item Adjust settings in the sidebar (optional)
        \item Click "Convert to Word Document"
        \item Download the generated .docx file
    \end{itemize}
\end{enumerate}

\section{Technical Limitations (Phase 1 Scope)}

\begin{enumerate}
    \item \textbf{Single-page documents only}: The current implementation is optimized for single-page English documents.
    
    \item \textbf{Formatting detection accuracy}: Bold and italic detection relies on heuristics and may not be 100\% accurate, especially with:
    \begin{itemize}
        \item Handwritten text
        \item Complex layouts
        \item Low-quality images
        \item Unusual fonts
    \end{itemize}
    
    \item \textbf{Language support}: Currently optimized for English text. PaddleOCR and EasyOCR support multiple languages, but formatting detection is tuned for English.
    
    \item \textbf{Alignment detection}: Simple margin-based approach may not work perfectly for all document layouts.
    
    \item \textbf{Image quality dependency}: OCR accuracy heavily depends on image quality, resolution, and contrast.
    
    \item \textbf{Model download}: First-time use of PaddleOCR or EasyOCR requires downloading models (\textasciitilde100MB each), which may take a few minutes depending on internet speed.
    
    \item \textbf{Processing time}: Deep learning OCRs (PaddleOCR, EasyOCR) are slower than Tesseract but provide better accuracy, especially for handwriting.
\end{enumerate}

\section{Best Practices for Users}

\begin{enumerate}
    \item \textbf{Image Quality}:
    \begin{itemize}
        \item Use high-resolution images (minimum 300 DPI recommended)
        \item Ensure good contrast between text and background
        \item Use clear, well-lit images
    \end{itemize}
    
    \item \textbf{Document Type}:
    \begin{itemize}
        \item Works best with printed or typed text
        \item Single-column layouts perform better
        \item Clear paragraph breaks improve structure detection
    \end{itemize}
    
    \item \textbf{Preprocessing}:
    \begin{itemize}
        \item Enable preprocessing for scanned documents
        \item Disable preprocessing for already-clear digital images
    \end{itemize}
    
    \item \textbf{Formatting Detection}:
    \begin{itemize}
        \item Enable formatting detection for documents with clear formatting
        \item Disable for simple text extraction if formatting detection causes issues
    \end{itemize}
\end{enumerate}

\section{Future Enhancements (Post-Phase 1)}

\begin{enumerate}
    \item Multi-page document support
    \item Improved formatting detection using machine learning
    \item Table detection and preservation
    \item Multiple language support with language selection
    \item Batch processing for multiple images
    \item Custom formatting rules configuration
    \item Export to other formats (PDF, HTML, etc.)
    \item OCR confidence threshold configuration
    \item Manual formatting correction interface
    \item Cloud storage integration
\end{enumerate}

\section{Testing Recommendations}

\begin{enumerate}
    \item \textbf{Test with various image types}:
    \begin{itemize}
        \item Scanned documents
        \item Digital screenshots
        \item Photos of documents
        \item Different resolutions
    \end{itemize}
    
    \item \textbf{Test formatting preservation}:
    \begin{itemize}
        \item Documents with bold text
        \item Documents with italic text
        \item Mixed formatting
        \item Different alignments
    \end{itemize}
    
    \item \textbf{Test edge cases}:
    \begin{itemize}
        \item Low-quality images
        \item Handwritten text (may not work well)
        \item Complex layouts
        \item Non-English text
    \end{itemize}
\end{enumerate}

\section{Conclusion}

Phase 1 successfully delivers a functional MVP of the Image-to-Word Converter with:
\begin{itemize}
    \item \checkmark \textbf{Multiple OCR engines} (PaddleOCR, EasyOCR, Tesseract)
    \item \checkmark OCR text extraction with high accuracy
    \item \checkmark Basic formatting detection (Bold, Italic)
    \item \checkmark Alignment preservation
    \item \checkmark Image preprocessing with multiple methods
    \item \checkmark Confidence threshold control
    \item \checkmark Clean, user-friendly interface
    \item \checkmark Production-ready code structure
    \item \checkmark Comprehensive error handling
\end{itemize}

The application is ready for testing and deployment, and can be extended in future phases based on user feedback and requirements.

\section{Deployment}

The application has been successfully deployed to Streamlit Cloud and is live at:

\textbf{Live Application URL}: \url{https://image-to-text-coneverter-ihgr5yvbz4237bq9ufyk8b.streamlit.app}

Users can access the application directly through this URL without any local installation required.

For detailed deployment instructions, see \textbf{DEPLOYMENT.md} in the project root.

\textbf{Quick Start - Streamlit Cloud (Recommended)}:
\begin{enumerate}
    \item Push code to GitHub
    \item Go to \url{https://share.streamlit.io}
    \item Connect your GitHub repo
    \item Deploy with one click!
\end{enumerate}

See \texttt{DEPLOYMENT.md} for other deployment options (Heroku, Docker, VPS, etc.).

\section{Contact \& Support}

For issues, questions, or contributions, please refer to the project repository at:
\url{https://github.com/Oyeumar8633/Image-to-Text-Coneverter}

\textbf{Live Application}: \url{https://image-to-text-coneverter-ihgr5yvbz4237bq9ufyk8b.streamlit.app}

\vspace{1cm}

\begin{center}
\textbf{Report Generated}: Phase 1 Implementation \\
\textbf{Status}: \checkmark Complete \\
\textbf{Version}: 1.0.0 \\
\textbf{Deployed}: \url{https://image-to-text-coneverter-ihgr5yvbz4237bq9ufyk8b.streamlit.app}
\end{center}

\end{document}
